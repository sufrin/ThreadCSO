\documentclass[12pt]{IOS-Book-Article-CPA-2017}
\usepackage{cso-paper,RCS}
%\SVN    $Id: cso-paper.tex 306 2017-12-19 16:29:29Z sufrin $
\def\AND{\mathtt{\&\&}}
% 
%%%%%%%%%%%%%%%%%%%%%%%%%%%%%%%%%%%%%%%%%%%%%%%%%%%%%%%%%%%%%%%%%%%%%%%%%%%%%%%%
%%%%%%%%%%%%%%%%%
%
% If you don't like endnotes then uncomment
%
%\long\def\note#1{\footnote{#1}}\let\listofendnotes=\relax
%

%%%%%%%%%%%%%%%%%%%%%%%%%%%%%%%%%%%%%%%%%%%%%%%%%%%%%%%%%%%%%%%%%%%%%%%%%%%%%%%%
\parindent=0pt\parskip=\smallskipamount

% Dec 28, 2014: textsc was not compatible with tabular; so I removed the tabular 


\begin{document}
\begin{frontmatter}
\setcounter{page}{1}
\title{Communicating Scala Objects\\ (2023 Revision)}
\runningtitle{Communicating Scala Objects (CSO Version 1.3, 2023 Revision)}
\runningauthor{Bernard Sufrin}

\author{\fnms{Bernard} \snm{SUFRIN}}
\address{
Oxford University Department of Computer Science\\
   Oxford OX1 3QD, England
   \\[4pt]
   {\small {\tt Bernard.Sufrin@cs.ox.ac.uk}}
}



\begin{abstract}
   In this paper we introduce the core features of CSO (Communicating
   Scala Objects) -- a notationally convenient embedding of the
   essence of \occam in a modern, generically typed, object-oriented
   programming language that is compiled to Java Virtual Machine
   (JVM) code.
      
   \noindent This revision describes version \textbf{1.3} of CSO as promulgated on
          \textbf{GitHub} in the repository \texttt{/sufrin/ThreadCSO}. It is
           compatible with \textbf{scala 2.13, and \textbf{Java 20}}. {The
           main notational difference is the use of \texttt{??[T]} and \texttt{!![T]}
           as abbreviations for \texttt{InPort[T]} and \texttt{OutPort[T]}. The
           doubling-up of \texttt{?} and \texttt{!} are a consequence of
           the deprecation of these identifiers in type names.           
          } 
   
\end{abstract}

\begin{keyword}
\textsf{occam} model\sep
concurrency\sep
Scala\sep
JCSP.%\sep
\end{keyword}
\end{frontmatter}


%%%%%%%%%%%%%%%%%%%%%%%%%%%%%%%%%%%%%%%%%%%%%%%%%%%%%%%%%%%%%%%%%%%%%%%%%%%%%%%%

\codestyle{scala}


\section*{Introduction}

On the face of it the Java virtual machine (JVM) is a very attractive platform for 
realistic concurrent and distributed applications and systems. On
the other hand, the warnings from at least parts of the ``Java establishment'' 
to neophyte Java programmers who think about using
threads are clear:

\begin{itemize}\parskip=\medskipamount\item[]
\begin{it}
If you can get away with it, avoid using threads. Threads can be
difficult to use, and they make programs harder to debug.

It is our basic belief that extreme caution is warranted when
designing and building multi-threaded applications
... use of threads can be very deceptive ...
in almost all cases they make debugging, testing, and maintenance
vastly more difficult and sometimes impossible. Neither the training,
experience, or actual practices of most programmers, nor the tools
we have to help us, are designed to cope with the non-determinism
...  this is particularly true in Java ...  we urge you to think
twice about using threads in cases where they are not absolutely
necessary ...\cite {jfc1}
\end{it}
\end{itemize}

But over the years a number of Java libraries \cite{dearsir,jcsp,jcsp2,ctj,ctjapp}
have demonstrated that the \occam programming model can be used
very effectively to provide an intellectually tractable \textit{discipline}
of concurrent Java programming that is harder to achieve by those
who rely on the lower level, monitor-based, facilities provided by
the Java language itself.

So in mid-2006, faced with teaching a new course on
concurrent and distributed programming, and wanting to make
it a \textit{practical} course that was easily accessible to
Java programmers, we decided that this was the way to go
about it.  We taught the first year of this course using a
Java library.\note{This was derived from an earlier library,
written in Generic Java, whose development had been inspired
by the appearance of the first public edition of JCSP.  The
principal differences between that library and the JCSP
library were the generically parameterized interfaces,
\textsf{InPort} and \textsf{OutPort} akin to
what JCSP called ``channel ends.''}

Our students' enthusiastic reaction to the \occam model was as
gratifying as their distaste for the notational weight of its
embedding in Java was dismaying. 
%
Although we discussed \textit{designs} for our concurrent programs
using a CSP-like process-algebra notation and a simplified form of
ECSP \cite{ecsp,ecspman}, the resulting \textit{coding gap} appeared
to be too much for most of the students to stomach.

At this point one of our visiting students introduced us to
Scala \cite{scala}, a modern object-oriented language that
generates JVM code, has a more subtle generic type system than 
Java, and has other features that make it very easy to construct
\textit{domain-specific languages} -- libraries that appear to be notational extensions.

After toying for a while with the idea of using Scala's Actor library
\cite{actors1,actors2}, we decided instead to develop a new Scala
library to implement the \occam model independently of existing
Java libraries,\note{Although Scala interoperates with Java, and
we could easily have constructed Scala ``wrappers'' for the JCSP
library and for our own derivative library, we wanted to have a
pure Scala implementation both to use as part of our instructional
material, and to ensure portability to the \textsf{.NET} platform
when the Scala \textsf{.NET} compiler became available.} and of
Scala's Actor library.\note{The (admirably ingenious) Actor library implementation is
complicated; its performance appears to scale well only for
certain styles of use; and it depends for correct functioning on a
global timestamp (\cite{actors2} p183).}
%
Our principal aim was to have a self-contained library we
could use to support subsequent delivery of our course (many
of whose examples are toy programs designed to illustrate
patterns of concurrency), but we also wanted to explore its
suitability for structuring larger scale Scala programs.

This paper is an account of the most important features of the core of 
the Communicating Scala Objects (CSO) library that emerged. We have assumed a little
familiarity with the conceptual and notational basis of \occam and
and some familiarity with Scala.



\begin{comment}
Readers familiar with JCSP and Scala may be able to get a quick initial 
impression of the relative notational weights of Scala+CSO and Java+JCSP 
by inspecting the definitions of \textsc{FairPlex} multiplexer components
defined on pages \pageref{scalafairplex} 
and \pageref{javafairplex} respectively. 
\end{comment}


%%%%%%%%%%%%%%%%%%%%%%%%%%%%%%%%%%%%%%%%%%%%%%%%%%%%%%%%%%%%%%%%%%%%%%%%%%%%%%%%%%%
%%%%%%%%%%%%%%%%%%%%%%%%%%%%%%%%%%%%%%%%%%%%%%%%%%%%%%%%%%%%%%%%%%%%%%%%%%%%%%%%%%%
%%%%%%%%%%%%%%%%%%%%%%%%%%%%%%%%%%%%%%%%%%%%%%%%%%%%%%%%%%%%%%%%%%%%%%%%%%%%%%%%%%%
\section{Processes}
A CSO process is a value with Scala type \textscala{PROC} and is
what an experienced object oriented programmer would call a
\textit{stereotype} for a thread. When a process is \textit{started}
any fresh threads that are necessary for it to run are acquired
from a pool; they are returned to the pool when the process
terminates.\note{The present pool implementation acquires new worker
threads from the underlying JVM when necessary and ``retires''
threads that have remained dormant in the pool for more than a
certain period.}

\subsection{Process notation}
\label{Process Notation}
Processes ($p:$ \textsc{PROC}) are first-class Scala values, denoted by
one of the following forms of expression:
\begin{itemize}\item[]
\begin{itemize}

\item[1a.]  \textsc{proc} \{ \(expr\) \}                   
\item[1b.]  \textsc{proc} (\(name: String\)) \{ \(expr\) \}
  
  A simple process (\(expr\) must be a command, \textit{i.e.} have type \textsc{Unit})
  
  If a $name$ is not given explicitly, one is automatically associated with the process value
  as it is constructed.

   
\item[2.]  \(p_1 \mathrel{||} p_2 \mathrel{||} ... \mathrel{||} p_n\)  
  
  \quad A parallel composition of $n$ processes (each \(p_i\) must have type \textsc{PROC})

\item[3.] \(\mathrel{||} collection\)        
  
  \quad Parallel composition of a finite collection of \textsc{PROC} values.
  
  \quad When $collection$ comprises \(p_1 ... p_n\) this is equivalent to  
  \(p_1 \mathrel{||} p_2 \mathrel{||} ... \mathrel{||} p_n\).
\end{itemize}
\end{itemize}


A frequently-occuring pattern of this latter form of
composition is one in which the collection is an iterated form, such as:
$\mathrel{||}$ \textscala{(for (i<-0 until n)} \textscala{yield} $p(i)$\textsc{)}.
This form denotes a process equivalent to: $p(0)\mathrel{||} p(1) \mathrel{||} ... \mathrel{||}p(n-1)$,

\subsection{Starting and running processes}
\label{Starting and running processes}
If $p$ is a process, then evaluation of the expression $p()$ runs the process.\note
{The expression $\mathsf{run}(p)$ has exactly the same effect as $p()$. 
The expression $\mathsf{fork}(p)$ runs $p$  in a new thread concurrent with the 
thread that invoked \textsf{fork}, and returns a \textit{handle} on the running process. The new thread is recycled when the process terminates.}
The following cases are distinguished:
%\note{A process $p$ may also be \textit{forked} using its \textsc{fork} method.}

\begin{enumerate}
\item[1a.] $p$ is \textsc{proc} \{ $expr$ \} \label{simple}
\item[1b.] $p$ is \textsc{proc} (\(name: String\)) \{ \(expr\) \}
\begin{itemize}
        \item[$\cdot$] $p()$ causes \{ $expr$ \} to be evaluated in the current thread (\textit{ie.} the thread that started the evaluation of $p()$).
        \item[$\cdot$] The process as a whole terminates when 
                       the evaluation of \{ $expr$ \} terminates
                       or throws an (uncaught) exception.
        
        \item[$\cdot$] The 
        behaviour of the expression \textscala{p()} 
        cannot be distinguished from 
        that of the expression \textscala{\{expr\}}, except that
        its $name$ is used to identify the thread running $p$ until
        $p$ terminates. This identification can be helpful when 
        inspecting a running CSO program using the CSO debugger.
\end{itemize}

\item $p$ is $p_1 \mathrel{||} p_2 \mathrel{||} ... \mathrel{||} p_n$ 
\begin{itemize}
        \item[$\cdot$] $p()$ causes all the processes $p_1 ... p_n$ to be run concurrently.
        \item[$\cdot$] Each of the processes except one is run in a new thread of its own;         
        the remaining process is run in the current thread.      
        \item[$\cdot$] The process as a whole terminates only  
                       when \textit{every} component  $p_i$ 
                       has terminated. But
              if one or more of the components
              terminated by throwing an uncaught exception then
              \textit{when and only when they have all terminated}
              these exceptions are bundled into a \textsc{ParException}
              which is re-thrown, \textit{unless they are all subtypes of} 
              \textsc{io.threadcso.process.Stopped}; in which case a single \textsc{io.threadcso.process.Stopped}
              is thrown.\note{This is
              because \textsc{io.threadcso.process.Stopped} exceptions signify anticipated
              failure, whereas other types of exception signify unexpected failure,
              and must be propagated rather than silently ignored. One useful
              consequence of the special treatment of \textsc{io.threadcso.process.Stopped}
              exceptions is explained in section \ref{Closing Ports
              and Channels}: \textit{Closing Ports and Channels}.}
              
\end{itemize}
\end{enumerate}



\begin{comment}
\subsection{Starting and running background processes}
It can sometimes simplify the construction of a large system to
have processes that run (more or less permanently) in the background.
%
If $p$ is a process, then evaluation of the expression $p~$\textscala{fork}
starts $p$ in a new thread that runs concurrently with
the current thread. The value of the expression is a  \textscala{process.Handle}
associated with the newly started thread and can be used to discover when
(and how) the thread terminates. 
\end{comment}

%%%%%%%%%%%%%%%%%%%%%%%%%%%%%%%%%%%%%%%%%%%%%%%%%%%%%%%%%%%%%%%%%%%%%%%%%%%%%%%%%%%
\section{Ports and Channels}

\subsection{Introduction}
A CSO channel has two ports, one at each end, and in general is intended to transmit
to its \textit{input port} the data that is written to its \textit{output port}. Ports
are parameterized by the type of data the channel transmits, and we define the abbreviations
\textscala{??[T]} and \textscala{!![T]} respectively for
\textscala{InPort[T]} and \textscala{OutPort[T]}.

The most important method of 
an \textscala{!![T]} is its write method
\begin{code}[]
    ! (value: T) 
\end{code}
and the most frequently-used methods of
an \textscala{??[T]} are its read method
\begin{code}[]
    ? () : T
\end{code}
and its \textit{read and evaluate} method 
\begin{code}[]
    ? [U] (f: T => U) : U 
\end{code}
The expression $port?(f)$ has exactly the same
effect as $f(port?())$, namely to read a datum from $port$ 
(waiting, if necessary, for one to become available) then apply 
the function $f$ to it.

The type \textscala{Chan[T]} is the interface implemented by all
channels that carry values of type \textsc{T}: it is declared by:
\begin{code}[]
    trait  Chan[T] extends InPort[T] with OutPort[T] { ... }
\end{code}

This makes \textscala{Chan[T]} a subtype of both \textscala{InPort[T]}
and \textscala{OutPort[T]}. It makes sense to think of a \textsc{Chan}
as embodying both an \textsc{InPort} and an \textsc{OutPort}.

The implicit contract of every conventional \textscala{Chan} implementation
is that it transmits the data written at its output port to its input
port in the order in which the data is written.  Different implementations
have different synchronization behaviours and different restrictions
on the numbers of processes that may access (\textit{i.e.} use the
principal methods of) their ports at any time. Channels may be
closed in various ways, in which case they
(eventually or immediately) cease to transmit data: 
see section \ref{Closing Ports and Channels} for a fuller
discussion of this.

The CSO core comes with several predefined channel implementations, the most
notable of which for our present purposes are:
\begin{itemize}
\item The \textit{synchronous} channels. These all synchronize
      termination of the execution of a ! at their output port with the
      termination of the execution of a corresponding ?  at their
      input port.\note{Other forms of synchronous channel, mostly now obsolete, are:
%%%%%%%%%%%%%%%%%%%%%%%%%%%%%%%%%%%%%%%%%%%%
\begin{itemize} 


\item \textscala{ManyOne[T]}  -- No more than one process at a time
       may access its input port; processes attempting to access its output
       port get access in nondeterministic order. {The
       name is a contraction of ``From \textsc{Many} possible writer
       processes to \textsc{One} reader process.'' The other
       forms of synchronous channel are named using the 
       same contraction convention.} 

\item \textscala{OneMany[T]}  -- No more than one process at a time
       may access its output port; processes attempting to access its input
       port get access in nondeterministic order. 

\item \textscala{ManyMany[T]} -- Any number of processes may attempt
       to access either port. Writing processes get access in nondeterministic
       order, as do reading processes.
      
\end{itemize}
%%%%%%%%%%%%%%%%%%%%%%%%%%%%%%%%%%%%%%%%%%%%
}

\begin{itemize}
\item \textscala{OneOne[T]}  -- no more than one process at a time
       may write to its output port or read from its input port.\note{The name is a
       contraction of ``From \textsc{One} writer process to \textsc{One}
       reader process.''} This is the classic \occam-style point
       to point channel. The channel stops transmitting data when it has been 
       closed for output or closed for input.
\item \textscala{N2N[T](writers: Int, readers: Int)} -- several 
      different processes at a time may write to its (shared) output port and
      likewise several may read from its (shared) input port.  
      Each value that is read is read by only one of the processes. The channel
      stops transmitting data when it has been closed for output $writers$ times
      or closed for input $readers$ times.
\end{itemize}


\item The \textit{buffered} channels:

\begin{itemize}
\item \textscala{OneOneBuf[T](n)} -- a one-to-one buffer of capacity
      $n$. It stops transmitting data when it has been closed for input, or
      when it has been closed for output and  no longer has any
      buffered data available to input.
\item \textscala{N2NBuf[T](n: Int, writers: Int, readers: Int)} --  a 
buffer of capacity $n$.  It stops transmitting data when
it has been closed for input $readers$ times, or when it has been closed for 
output $writers$ times and  no longer has any
buffered data available to input.
\end{itemize}
\end{itemize}
\noindent Access restrictions are enforced by a combination of:
\begin{itemize}
\item Type constraints that permit sharing requirements to be enforced statically.
\begin{itemize}
\item All output port implementations that support shared access
      have types that are subtypes of \textsc{SharedOutPort}.

\item All input port implementations that support shared access
      have types that are subtypes of \textsc{SharedInPort}.

\item All channel implementations that support shared access
      to both their ports have types that are 
      subtypes of \textsc{SharedChannel}.

\item Abstractions that need to place sharing requirements on
      port or channel parameters do so by declaring them with the appropriate 
      type.\note{See, for example, the component \textsc{mux2} defined in
      \Listing \ref{mux}.}
      
\end{itemize}
\item Run-time checks that offer \textit{partial} protection against
deadlocks or data loss of the kind that can could
otherwise happen if unshareable ports were
inadvertently shared. 
\begin{itemize}
\item If a read is attempted from a channel with an unshared input port 
      before an earlier read has terminated, then an illegal state
      exception is thrown.
\item If a write is attempted to a channel with an unshared output port 
      before an earlier write has terminated, then an illegal state
      exception is thrown.
\end{itemize}
These run-time checks are limited in their effectiveness because it is
might be possible for a single writer process to work fast enough to
satisfy illegitimately sharing reader processes without being
detected by the former check, and for the dual situation
to remain undetected by the latter check.
\end{itemize}



\subsection{Examples}


In \Listing \ref{jcspnet} we show how to connect a sequence 
of $n$ producers to a sequence of $n$ consumers using a single
multiplexed channel that carries values accompanied by 
the index of their producer to a demultiplexer that dispatches these
values to the corresponding consumer.
Readers familiar with JCSP may find it useful to compare this
with the network illustrated in section 1.5 of \cite{jcsp2}.

\begin{code+}[...]{float,frame=lrtb,label=jcspnet,caption={A network of producers connected to consumers by a multiplexed channel}}
    def producer(i: int, !![T]) : PROC = ...
    def consumer(i: int, ??[T]) : PROC = ...
    
    def mux[T] (ins: Seq[??[T]],   out: !![(int, T)]) : PROC = ...
    def dmux[T](in:  ??[(int, T)], outs: Seq[!![T]])  : PROC = ...
    
    
    val left, right = 
        for (_ <- 0 until n) yield OneOne[T] // 2 arrays of n channels
    val mid = OneOne[(int, T)]               // a channel
    
    (  || (for (i<-0 until n) yield producer(i, left(i)))
    || mux(left, mid)
    || dmux(mid, right)
    || || (for (i<-0 until n) yield consumer(i, right(i)))
    )()  
\end{code+}

As observed in that paper this isn't the most efficient
way of connecting the producers to the consumers within a single
JVM; and in \Listing \ref{jcspnet1} we show a network in which producers
and consumers are connected directly.

\begin{code+}{float,frame=lrtb,label=jcspnet1,caption={A network in which producers are connected directly to consumers}}
    def producer(i: int, !![T]) : PROC = ...
    def consumer(i: int, ??[T]) : PROC = ...
    
    val con = for (_ <- 0 until n) yield OneOne[T]
    
    (  || (for (i<-0 until n) yield producer(i, con(i)))
    || || (for (i<-0 until n) yield consumer(i, con(i)))
    )()  
\end{code+}

The signatures of the components \textsc{producer}, \textsc{consumer},
\textsc{mux}, and \textsc{dmux} in \Listings \ref{jcspnet} and
\ref{jcspnet1} specify the types of port (channel end) they require;
but the subtype relation between channels and ports means that when
connecting these components we can simply provide the connecting
channels as parameters, and that the components take the required
\textit{view}s of them. This means we needn't name the ports
explicitly, and significantly reduces the degree of \textit{formal
clutter} in the network description.\note{The reduction of formal
clutter comes at the cost of forcing readers to refer back to 
the component signatures to ascertain which
ports they actually use. The JCSP designers made the tradeoff in 
the other direction.}

In \Listing \ref{mux}  we show how to implement two (unfair) 
multiplexers and a demultiplexer of the kind that might have
been used in \Listing \ref{jcspnet}.\note{We have used
the plain form of read $(mid?())$ and its read-and-evaluate form $(ins(i) ? { v \Rightarrow mid!(i, v)})$ 
simply to give an example of the latter.}

\begin{code+}[...]{float,label=mux,frame=tblr,caption={Two multiplexers and a demultiplexer}}
  def mux1[T] (ins: Seq[??[T]], out: !![(Int, T)]) : PROC =
  { val mid = N2N[(Int, T)](0, 1) // Many writers; one reader
    (  proc { while(true) { out!(mid?()) } } ||
    || (for (i<-0 until ins.length) yield 
            proc { while(true) ins(i) ? { v => mid!(i, v)} })
    )
    
  }   
   
  def mux2[T] (ins: Seq[??[T]], out: SharedOutPort[(Int, T)]) : PROC =
      || (for (i<-0 until ins.length) yield 
              proc { while (true) {out!(i, ins(i)?())} })   

  def dmux[T](in:  ??[(Int, T)], outs: Seq[!![T]]) : PROC = proc { 
    while (true) { val (n, v) = in?(); outs(n)!v  } 
  }
\end{code+}

A multiplexer process generated by \textscala{mux1} is the concurrent composition 
of a collection of ``labelling'' processes, each of which outputs labelled copies of its
input, via an \textscala{N2N[(Int,T)]} channel, to a forwarding process that
writes them to the \textsc{out} port. The forwarding process is necessary
because the type-signature of \textsc{mux1} does not constrain the
kind of port that is passed to it as a parameter, so in programming \textsc{mux1}
we must assume that that the port is not shareable.

On the other hand, \textscala{mux2} requires that its \textscala{out} 
parameter is shareable, so it composes a collection of labelling processes that
write directly to \textsc{out}.

The function \textsc{dmux} generates demultiplexer processes that forward
labelled inputs to the appropriate output ports.



%%%%%%%%%%%%%%%%%%%%%%%%%%%%%%%%%%%%%%%%%%%%%%%%%%%%%%%%%%%%%%%%%%%%%%%%%%%%%%%%%%%
\section{Extended Rendezvous} 

\subsection{Introduction}
As we explained earlier, the \textit{synchronous} channel implementations 
ensure that \textit{termination} of a write (!) at their output port is
synchronized with the termination of the corresponding read (?) at
their input port. 
%
Although a standard read (or read-and-evaluate) terminates once the data is transferred
between the writer and the reader process, an \textit{extended
rendezvous read} specifies that a computation on the transferred data is to take place \textit{in the
reader process}. It is only when this computation terminates
that the read is considered to have terminated and the writing process
is permitted to proceed.

The usual form of an extended rendezvous read from \textsc{in: ??[T]} is\note{The
most general form of extended rendezvous read is \textsc{in??f} where \textsc{f} denotes
a function of type \textsc{T=>U}. The type of \textsc{in??f} is then \textsc{U}.}
\begin{code}
        in ?? { bv => body } 
\end{code}
It is evaluated by transferring a value, $v$, from the process at
the output end of the channel (if necessary waiting for one to
become ready), then applying the (anonymous) function \textsc{\{ bv => body \}}
to $v$. The read is considered to have terminated when
this application has been completely evaluated. At this point the writing process 
is permitted to proceed and the result of the application is returned from the read. 


\begin{comment}
In CSO the notation for an extended rendezvous read from \textsc{in: ??[T]} is
\textsc{in ? f}, where \textsc{f: T=>U} denotes function of type \textsc{T=>U}. 
The expression \textsc{in ? f} has type \textsc{U}, and it is evaluated by 
transferring a value, $v$, from the process at the output end of the
channel (if necessary waiting for one to become ready), then evaluating \textsc{f}($v$).
The read is considered to have terminated when the evaluation of \textsc{f}($v$)
is complete;  at which point the writing process proceeds.
It is more readable to express the function
anonymously than to name it; \textit{i.e.} to express the whole rendezvous in the form 
\textsc{in ? \{v => body\}} rather than in the form \textsc{\{def f(v)=body; in ? f\}}.
\end{comment}


\subsection{Example: monitoring interprocess traffic}

An easily understood rationale for extended rendezvous is given in
\cite{jcsp}. We are asked to consider how to monitor the interprocess
traffic between a producer process connected to a  consumer process
via a simple channel \textit{without interfering with producer-consumer
synchronization}.
%
We want to construct a process that is equivalent to
\begin{code+}[...]{}
        { val mid = OneOne[T]
          producer(mid) || consumer(mid)
        }
\end{code+}
but which also copies traffic on mid to a monitor process of
some kind.

A first approximation to such a process is
\begin{code+}[...]{}
        { val left, mon, right = OneOne[T]
          (  producer(left) 
          || proc { repeat { val v = left?(); mon!v; right!v }
          || consumer(right)
          || monitor(mon)
          )
        }
\end{code+}
But this interferes with producer-consumer synchronization,
because once \textsc{left?()} has been executed, \textsc{producer}
is free to proceed. More specifically, it is free to proceed before
\textsc{consumer} reads from \textsc{right}. If the context in which
this network of process runs is {tolerant} of an additional
degree of buffering this is not problematic; but if it is not,
then we need to be able to synchronize the read from \textsc{right}
with the write to \textsc{left}.

The problem is solved by replacing the body of the copying process
\begin{code+}[...]{}
          { val v = left?(); mon!v; right!v }
\end{code+}
with a body in which the outputs to \textsc{mon} and \textsc{right} 
are part of an extended rendezvous with the producing process, namely:
\begin{code+}[]{}
          { left ?? { v => {mon!v; right!v} } }
\end{code+}

The extended rendezvous is executed by reading a value from \textsc{left}, then
applying the function \textsc{\{ v => \{mon!v; right!v\} \}} to it.
Termination of the write to \textsc{left} is synchronized with termination
of the evaluation of the function body, so the producer writing to \textsc{left}
cannot proceed until the consumer has read from \textsc{right}.

The extended rendezvous doesn't terminate until \textsc{\{mon!v; right!v\}}
has terminated, but delays the output to \textsc{right} until
the output to \textsc{mon} has terminated. The following reformulation
relaxes the latter  constraint, thereby removing a potential source of
deadlock:
\begin{code+}[...]{}
    { left ?? { v => {(proc{mon!v} || proc{right!v})()} } }
\end{code+}

\noindent It is a simple matter to abstract this into a reusable component:
\begin{code+}[...]{}
    def tap[T](in: ??[T], out: !![T], mon: !![T]) =
        proc
        { repeat { in ? { v => {(proc{mon!v} || proc{out!v})()} } } }
\end{code+}

\subsection{Example: simplifying the implementation of synchronous inter-JVM channels}

Extended rendezvous could also be used to good effect in the
implementation of synchronized inter-JVM or cross-network connections,
where it can keep the overt intricacy of the code manageable. Here we
illustrate the essence of the implementation technique, which employs
the two ``network adapter'' processes.

\begin{code*}[netstuff.scala]
import io.threadcso._
object netstuff
{
\end{code*}
\begin{code+}{}
    def copyToNet[T](in: ??[T], net: !![T], ack: ??[Unit]) =
        proc { repeat { in ?? { v => { net!v; ack?() } } } }
\end{code+}
and
\begin{code+}[netstuff.scala]{}
    def copyFromNet[T](net: ??[T], ack: !![Unit], out: !![T]) =
        proc { repeat { out!(net?()); ack!(()) } }
\end{code+}

The effect of using the extended rendezvous in \textsc{copyToNet}
is to synchronize the termination of a write to \textsc{in} with
the reception of the acknowledgement from the network that the
value written has been transmitted to \textscala{out}. 


At the producer end of the connection, we set up a bidirectional
network connection that transmits data and receives 
acknowledgements. Then we connect the producer to the
network via the adapter:

\begin{code+}[...]{}
    def producer(out: !![T]) = ...    
    val (toNet, fromNet): (!![T], ??[Unit]) = ...
    val left = OneOne[T]    
    ( producer(left) || copyToNet(left, toNet, fromNet) )()
\end{code+}

At the consumer end the dual setup is employed
\begin{code+}[...]{}
    def consumer(in: ??[T]) = ...   
    val (toNet, fromNet): (!![Unit], ??[T]) = ...
    val right = OneOne[T]    
    ( copyFromNet(fromNet, toNet, right) || consumer(right) )()
\end{code+}

\begin{code*}[netstuff.scala]
    def main(args: Array[String]) =
    { import io.threadcso.component._
      val left, net, right = OneOne[String]
      val ack  = OneOne[Unit]
      (  proc { for (arg<-args) left!arg } || 
         copyToNet(left, net, ack)         ||
         copyFromNet(net, ack, right)      ||
         console(right)
      ) ()    
    }
}
\end{code*}


%%%%%%%%%%%%%%%%%%%%%%%%%%%%%%%%%%%%%%%%%%%%%%%%%%%%%%%%%%%%%%%%%%%%%%%%%%%%%%%%%%%
\section{Closing Ports and Channels (clean termination)}
\label {Closing Ports and Channels}
\subsection{Introduction}

A port may be \textit{closed} at any time, \textit{including
after it has been closed}. The trait \textsc{InPort} has method
\begin{code}[...]
        closeIn: Unit
\end{code}
whose invocation embodies a promise \textit{on the part of its invoking thread}
never again to read from that port. Once it has been invoked, the method $canInput$
will always yield false for that port.
%
Similarly, the trait \textsc{OutPort} has method
\begin{code}[]
        closeOut: Unit
\end{code}
whose invocation embodies a promise \textit{on the part of its invoking thread}
never again to write to that port. Once it has been invoked, the method $canOutput$
will always yield false for that port.

It can sometimes be appropriate to \textit{forbid} a channel
to be used for further communication, and the \textsc{Chan} trait 
has an additional method for that purpose, namely:
\begin{code}[]
        close: Unit
\end{code}
The important design questions that must be considered are: 
\begin{enumerate}
\item   What happens to a process that attempts, or is 
        attempting, to communicate through a port whose peer
        port is closed, or which closes during the attempt?

\item   What does it mean to close a \textit{shared} port?
\end{enumerate}
Our design can be summarised concisely; but we must first
explain what it means for a {channel} to be closed:

\begin{quote}
        \textit{Definition: }A channel is \textit{closed} if it has
        been closed by enough calls of \textsc{closeOut} at its \textsc{OutPort} 
        or by enough calls of \textsc{closeIn} at its \textsc{InPort}, or by
        a call of its \textsc{close} method.
\end{quote}
\begin{center}
\begin{tabular}{|l|c|c|c|}\hline
\multicolumn{1}{|c|}{Channel}& \multicolumn{2}{c|}{Enough} &\multicolumn{1}{c|}{Close takes}\\
\multicolumn{1}{|c|}{type}& closeOut & closeIn & \multicolumn{1}{c|}{effect on readers}\\\hline
OneOne          & 1              & 1             & immediately\\
OneOneBuf($n$)       & 1              & 1             & when drained\\
N2N($n$, $writers$, $readers$) & $\widehat{writers}$ & $\widehat{readers}$  & immediately\\
N2NBuf($n$, $writers$, $readers$) & $\widehat{writers}$ & $\widehat{readers}$  & when drained\\\hline
ManyOne & $\infty$ & 1 & immediately\\
OneMany & 1 & $\infty$ & immediately\\
ManyMany & $\infty$ & $\infty$ & never\\\hline
\end{tabular}
\end{center}

The table above summarises what we mean by ``enough'' -- using the notation $\widehat{num}$
to mean $\infty$ when $num=0$ and $num$ otherwise. For example
an \textsc{N2N} channel specified with $writers>0$, and $readers>0$  closes
after either $writers$ \textscala{closeOut} calls or $readers$
\textscala{closeIn} calls; but if $writers=0$ then any number of calls
of \textscala{closeOut} can be made without the channel closing, and
if $readers=0$ then any number of calls of \textscala{closeIn} can
be made without the channel closing.

The rationale for this is that shared ports are used as
``meeting points'' for senders and receivers, and that the fact
that one sender or receiver has undertaken never to communicate
should not necessarily result in the right to do so being denied to others.\note{This
is a deliberate choice, designed to keep shared channel semantics
simple. More complex channel-like abstractions -- such as one 
in which a non-shared end is informed when all subscribers to 
the shared end have disappeared -- can always be layered on top of it.}

The effects of closing ports and/or channels now can be summarised as follows:

\begin{itemize}
\item Writer behaviour
\begin{enumerate}
\item An attempt to write to a closed channel raises the exception
      \textsc{Closed} in the writing thread.
\item Closing a channel whose OutPort is waiting
      in a write raises the exception
      \textsc{Closed} in the writing thread.
\end{enumerate}    

\item Reader behaviour
\begin{enumerate}
\item An attempt to read from a closed channel raises the exception
      \textsc{Closed} in the reading thread. If the channel 
      is buffered then this exception is raised
      only once the last remaining buffered value has been read.
\item Closing a channel whose InPort is waiting
      in a read raises the exception
      \textsc{Closed} in the reading thread.
\end{enumerate}
\end{itemize}


\subsection{Termination of networks and components}
The \textsc{Closed} exception is one of a family of runtime exceptions,
the \textsc{Stop} exceptions, that play a special role in ensuring 
the clean termination of networks of communicating processes.

The form
%\begin{alltt}
        \textsc{repeat} (\(expr\sb{guard}\)) \{ \(expr\sb{body}\) \}
%\end{alltt}
behaves the same as
%\begin{alltt}
        \mbox{\textsc{while} (\(expr\sb{guard}\)) \{ \(expr\sb{body}\) \}}
%\end{alltt}
except that the raising of a \textsc{Stop} exception during the
execution of the \(expr\sb{body}\) causes it 
to terminate normally.
%
The  form
%\begin{alltt}
        \textsc{repeat} \{ \(expr\sb{body}\) \}
%\end{alltt}
is equivalent to 
%\begin{alltt}
        \textsc{repeat} (\textsc{true}) \{ \(expr\sb{body}\) \}
%\end{alltt}

The behaviour of \textsc{repeat} simplifies the description of cleanly-terminating iterative components
that are destined to be part of a network. For example, consider
the \textsc{copy} component of \Listing \ref{copy}, which has an iterative 
copying phase followed by a close-down phase.
\begin{code+}[...]{float,label=copy,caption={A terminating copy component}}
    def copy[T](in: ??[T], out: !![T]) = 
    proc {
      repeat { out!(in?()) }    // copying
      out.closeOut; in.closeIn  // close-down
    }
\end{code+}
It is evident that the copying phase terminates if the channel
connected to the input port is closed before that connected to the
output port. Likewise, if the channel connected to the output port
is closed before (or within) a write operation that is attempting
to copy a recently-read datum. In either case the component moves
into its close-down phase, and this results in one of the 
channels being closed again while the other is closed anew.
In nearly all  situations this behaviour is satisfactory, but it is
worth noticing that it can result in a datum being silently lost (in the implicit buffer between 
the \textsc{in?()} and the \textsc{out!}) when a
network is closed from ``downstream''.\note{\textit{i.e.} from the \textsc{out} 
direction. On the face of it
it looks like this could be avoided by  reprogramming the component
with a stronger guard to the iteration, \textit{viz} as:
\textsc{repeat (out.canOutput) \{ out!(in?()) \}}
\textit{but this is not so}, because the \textsc{out.canOutput} test and
the \textsc{out!} action are not joined atomically, so the 
channel associated with the 
output port could be closed between being polled in the guard and
being written to in the body of the loop.}
%\note
%{Solving this particular problem properly will require the implementation of 
%a form of guarded output alternation, but we have yet to investigate output guards.}

In section {\ref{Starting and running processes}} we explained that
on termination of all the components of a concurrent process: 
(a) if they all terminated normally then the concurrent process 
itself terminates normally; (b) if all components that terminated abnormally
terminated with a \textsc{Stop} exception then the concurrent process 
itself terminates by throwing a \textsc{Stop} exception; (c) otherwise the concurrent process terminates 
by throwing a \textsc{ParException}.

One consequence of (b) is that it is relatively simple to arrange
to reach the closedown phase of an iterated component that does
concurrent reads and/or writes. 
%
For example, the \textsc{tee} component of \Listing \ref{tee} broadcasts
data from its input port to all its output ports concurrently: if
the input port closes, or if any output port is closed 
before or during a broadcast, then the component stops
broadcasting and closes all its ports.\note{We observe, without much enthusiasm, that there is no
particular constraint on the order in which the ports are closed, so they could be closed
in parallel by the more complicated:

\textsc{(|| (for (out<-outs) yield proc { out.closeOut }) || in.closeIn)()}

}

\begin{code+}[...]{float,label=tee,caption={A data broadcasting component}}
   def tee[T](in: ??[T], outs: Seq[!![T]]) =
   proc { var data      = in.nothing  // unspecified initial value
          val broadcast = || for (out<-outs) yield proc { out!data }
          repeat { in ?? { d => { data=d;  broadcast() }}}
          for (out<-outs) out.closeOut; in.closeIn
        }
\end{code+}

This is because closing \textsc{in} results in a \textsc{Closed} exception 
being thrown at the next \textsc{in??}; and because closing an
output port causes the corresponding \textsc{out!data} to 
terminate by throwing a \textsc{Closed}, which is
propagated in turn by the \textsc{||} when it terminates.\note{Although it is incidental to the theme of this example,
it is worth noticing that we construct the concurrent process
\textsc{broadcast} before starting the iteration. 
While this is not strictly necessary, it provides an improvement in
efficiency over:
\textsc{repeat \{ in ? \{ d => \{|| (for (out<-outs) yield proc \{ out!d \})() \}\}\}}.
This is  because the expression: 
\textsc{||(for (out<-outs)} ... \textsc{)}  that constructs the
concurrent broadcast process is
evaluated only once, rather than being evaluated once per broadcast.}

Careful programming of the closedown phases of communicating components
is needed in order to assure the clean termination of networks
of interconnected processes, and this is facilitated by the \textsc{Stop}-rethrowing behaviour
of $||$, and the behaviour of \textsc{repeat} when its body \textsc{Stop}s.
%%%%%%%%%%%%%%%%%%%%%%%%%%%%%%%%%%%%%%%%%%%%%%%%%%%%%%%%%%%%%%%%%%%%%%%%%%%%%%%%%%%
\section{Alternation}
\label{Input Alternation}
\subsection{Input-guarded events}
%Input alternations are first class values of type \textsc{Alt}.
%
Alternation constructs enable an input or output action to be
performed after being selected from those that are ready on one or more ports. 
The simplest form of an \textsc{alt} consists of a collection of 
\textit{guarded events}:\note{Guard expressions must be free of side-effects,
and a $(guard)$ that is literally \textsf{(true)} may be omitted. }

\begin{alltt}
        \textsf{\textbf{alt}} ( \( (guard\sb1 \AND port\sb1) \) \texttt{=?=>} \{ \(bv\sb1\) \texttt{=>} \(cmd\sb1\) \}
           | ...
           | \( (guard\sb{n} \AND port\sb{n}) \) \texttt{=?=>} \{ \(bv\sb{n}\) \texttt{=>} \(cmd\sb{n}\)\}
           )
\end{alltt}
%
An input event of the form \((guard \AND port)\) ~~\verb/=?=>/~~ \{ \(bv\) \verb/=>/ \(cmd\) \} 
\begin{itemize}
\item is said to be \textit{enabled}, if $port$ is open and $guard$ evaluates to true
\item is said to be \textit{ready} if $port$ is ready to read
\item is \textit{fired} by reading $port$, binding the value read to $bv$ and then executing $cmd$.
\end{itemize}
The execution of an \textscala{alt} proceeds
in principle\note{We say ``in principle'' because we wish to retain
the freedom to use a much more efficient implementation than is
described here, namely an adaptation of that described in \cite{BGJK}.} in phases as follows:
\begin{enumerate}
\item All the event guards are evaluated, and then
\item The current thread waits until (at least one) enabled event is ready, and then
\item One of the ready events is chosen and fired.
\end{enumerate} 

If no events are enabled after phase 1, or if all the channels
associated with the ports close while waiting in phase 2, then the
\textsc{Abort} exception (which is also a form of \textsc{Stop}
exception) is raised.

If $evs$ is a collection of guarded events, then \textsc{serve}$(evs)$  executes these
phases repeatedly (until a \textsc{Stop} exception is thrown), but the choices made in phase 3 are made in such
a way that if the same group of guards turn out to be ready during
successive executions, they will be fired in turn.

For example, the method \textsc{tagger} below constructs a tagging
multiplexer that ensures that neither of its input channels
gets too far ahead of the other. The tagger terminates
cleanly when its output port is closed, or if both its
input channels have been closed. 
%{frame=tlbr,label=tagger,caption={a tagging multiplexer}}
\begin{code+}[...]{}
  def tagger[T](l: ??[T], r: ??[T], out: !![(Int, T)]) = 
  proc
  { var diff = 0
    serve ( (diff < 5  && l) =?=> { x => out!(0, x); diff+=1 }
          | (diff > -5 && r) =?=> { x => out!(1, x); diff-=1 }
          ) 
    repeat { out!(0, l?()) } 
    repeat { out!(1, r?()) } 
    l.closeIn; r.closeIn; out.closeOut 
  }
\end{code+}
Notice that the  \textsc{serve} will also terminate if the ``wrong'' channel closes (for
example $l$ if $diff<5$); but following such termination at most one of the subsequent  
\textsc{repeat}s
(the ``right'' one) can perform a successful read and write.


A \textsc{prialt} is formed in the same way as an \textsc{alt},
and is executed in nearly the same way, but the choice of which among
several ready guards to fire always favours the earliest in the sequence.
A \textsc{priserve} repeats a \textsc{prialt}.

\subsection{Output-guarded events}
In late 2008 the \textit{outport guard} notation was added to CSO. Its simplest
form is exemplified in the following implementation of a merging component 
that buffers no more than 20 inputs tagged with sequence numbers, and 
outputs them when there is demand from \verb/out/. The serve loop 
will terminate when none of the guarded events can occur -- after which 
the three ports are closed.
\begin{code+}[...]{}
  def taggedMerge[T](l: ??[T], r: ??[T], out: !![(Int, T)]) = 
  proc
  { var seqn = 0 // sequence number
    var nbuf = 0 // number buffered 
    val q        = scala.collection.mutable.Queue[(Int, Int, T)]
    serve ( (nbuf<20 && l)   =?=> { x => q.enqueue((seqn+=1, 0, x)); nbuf+=1; }
          | (nbuf<20 && r)   =?=> { x => q.enqueue((seqn+=1, 1, x)); nbuf+=1; }
          | (nbuf>0  && out) =!=> { nbuf-=1; q.dequeue } 
          ) 
    l.closeIn; r.closeIn; out.closeOut  
  }
\end{code+}
The most general form of output-guarded event is

\begin{center}
\((guard \AND port)\) ~~\verb/=!=>/~~ \{ expression \} \verb/==>/ \{ \(cmd\) \} 
\end{center}


It is \textit{ready} if its guard is true and $port$ is ready to be written.
It is \textit{fired} by evaluating $expression$ (which can be a sequential composition) 
and writing its value ($v$, say) to $port$.\note{An
early version of CSO provided an even more general form, in which
the epilogue was a function, and the value of the expression was passed to
this function once it had been transmitted. This proved incompatible with
the Scala type system.}  
When it has been written $cmd$ (known as the \textit{epilogue}) is executed.\note
{The operator $\Longrightarrow$ that introduces the {epilogue} is pronounced ``and then''.}

If there is no need for an epilogue, the event can be written:
\begin{center}
\((guard \AND port)\) ~~\verb/=!=>/~~ \{ expression \}
\end{center}
The final guard of the \verb/taggedMerge/ serve loop could have been written with an epilogue, by
doing the buffer-size accounting after the dequeued datum has been transmitted.
\begin{code+}[...]{}
          | (nbuf>0  && out) =!=> { q.dequeue } ==> { nbuf-=1; }
\end{code+}

\subsection{Collections of guards}
Alternations can
be composed of collections of guards, as illustrated 
by the fair multiplexer defined below.\note{It is perhaps worthwhile comparing this construction
with that of the analogous JCSP component shown in 
\Listing \ref{javafairplex} (page \pageref{javafairplex}).}
\begin{code*}[fairPlex.scala]
import ox.CSO._
object fairPlex
{
\end{code*}
%literate={<-}{$\leftarrow$}1{==>}{$\Longrightarrow$}2
%{frame=lrtb,label=scalafairplex,caption=Fair Multiplexer: CSO}
\label{scalafairplex}
\begin{code+}{}
  def fairPlex[T](ins: Seq[??[T]], out: !![T]) = 
      proc { serve (| (for (in <- ins) yield in =?=> { t => out!t })) }
\end{code+}
\begin{code*}[]
}
\end{code*}

They can also be composed by combining collections and single guards. For example,
the following is an extract from a multiplexer that can be
dynamically set to favour a specific range of its input ports.
It gives priority to its range-setting channels.
\begin{code*}[minmax.scala]
import ox.CSO._
object minmax
{ 
\end{code*}
\begin{code}
  def primux[T](MIN: ??[Int], MAX: ??[Int], ins: Seq[??[T]], out: !![T]) = 
  proc
  { var min = 0
    var max = ins.length - 1
    priserve ( MIN =?=> { n => min = n }
             | MAX =?=> { n => max = n }
             | | (for (i <- 0 until ins.length) yield 
                    (max>=i && i>=min && ins(i)) =?=> { t => out!t } )
             ) 
  }
\end{code}
\begin{code*}[]
}
\end{code*}



\subsection{Timed Alternation}
\label{Timed Alternation}
An alternation may be qualified with a deadline and code to be
executed in case of a timeout.\note{The implementation of
this feature employs a nonzero
timeout for the wait in phase 2, and is not subject to any
potential race conditions.} We illustrate this feature
with an extended example that defines the transmitter and receiver
ends of an inter-JVM buffer that piggybacks ``heartbeat'' confirmation
to the receiving end that the transmitting end is still alive.

First we define a Scala type \textsc{Message} whose values are of one
of the forms \textsc{Ping} or \textsc{Data(v)}. 
\begin{code}
trait Message
case  object  Ping             extends Message {}
case  class   Data[T](data: T) extends Message {}
\end{code}

The transmitter end repeatedly forwards data received from \textsc{in}
to \textsc{out}, but intercalates \textsc{Ping} messages whenever
it has not received anything for \textsc{pulse} nanoseconds.\note{Nanosecond
is now the unit of resolution of CSO time. It's not yet realistic
to measure delays or timeouts in small numbers of nanoseconds, so
appropriate multipliers, such as \textsc{microSec, milliSec, Sec, Min, Hour, Day}
are provided as part of the CSO package.}
\begin{code}
def transmitter[T](pulse: Nanoseconds, in: ??[T], out: !![Message]) = 
  proc 
  { serve (in =?=> {x=>out!Data(x)} | after(pulse) ==> { out!Ping }) } 
\end{code}

The receiver end (whose deadline should be somewhat larger 
than the transmitter's pulse) repeatedly reads from \textsc{in},
discarding \textsc{Ping} messages and forwarding ordinary
data to \textsc{out}. If (in each iteration) a message has not been
received before the current deadline, the receiver backs off a little more, 
but eventually a message is sent to the \textsc{fail} channel.

\begin{code}
 def receiver[T](pulse: Nanoseconds, in: ??[Message], out: !![T], fail: !![Unit]) = 
 proc 
 { var backoff = 10
   serve( in =?=> { case Ping      => backoff = (backoff+1) % 10
                   case Data(d:T) => out!d; backoff = (backoff+1) % 10}
        | after(pulse+pulse/backoff) ==> 
                 { if (backoff==1) fail!() else backoff -=1 }
        )
 }
\end{code}

Though timeout is cheap and safe to implement, the
technique used above may not be suitable for use in components where
there is a need for more subtle interplay between timing and channel
input.
%
But such components can always be constructed (and in a way that
may be more familiar to \occam programmers) by using periodic timers,
such as the simple one shown in \Listing
\ref{periodictimer}.

For example, \Listing \ref{transmitter2} shows the definition of
an alternative transmitter component that ``pings'' if the periodic
timer ticks twice without an intervening input becoming available
from \textsc{in}, and ``pongs'' every two seconds regardless of
what else happens.

\begin{code+}[...]{float=h,frame=tlbr,label=transmitter2,caption={A conventionally-programmed transmitter}}
 def transmitter2[T](pulse: Nanoseconds, in: ??[T], out: !![Message]) = 
 proc 
 { val tick  = periodicTimer(pulse)
   val tock  = periodicTimer(2*Sec)
   var ticks = 0
   priserve ( tock =?=> { case () => out!Pong }
            | in   =?=> { case t  => out!Data(t); ticks = 0 }
            | tick =?=> { case () => ticks+=1; if (ticks>1) out!Ping }
            ) 
   tick.close
   tock.close
 } 
\end{code+}
In the periodic timer of \Listing \ref{periodictimer} the \textsc{fork} method of a process
is used to start a new thread that runs concurrently with the current thread and periodically writes
to the channel whose input port represents the timer. Closing the
input port terminates the \textsc{repeat} the next time the interval
expires, and thereby terminates the thread.
\begin{code+}[...]{float=hb,frame=tlbr,label=periodictimer,caption={A simple periodic timer}}
 def periodicTimer(interval: Nanoseconds) : ??[Unit] = 
 { val chan = OneOne[Unit]
   proc { repeat { sleep(interval); chan!() } } . fork
   return chan
 }
\end{code+}

\subsection{Restrictions on alternation}
For reasons of efficiency and to keep implementations simple \textit{at most
one port} of a channel may participate in an alternation 
construct at any one time.\note{Gavin Lowe \cite{lowe} worked with an earlier
version of CSO to remove this restriction. }


%%%%%%%%%%%%%%%%%%%%%%%%%%%%%%%%%%%%%%%%%%%%%%%%%%%%%%%%%%%%%%%%%%%
%%%%%%%%%%%%%%%%%%%%%%%%%%%%%%%%%%%%%%%%%%%%%%%%%%%%%%%%%%%%%%%%%%%
%%%%%%%%%%%%%%%%%%%%%%%%%%%%%%%%%%%%%%%%%%%%%%%%%%%%%%%%%%%%%%%%%%%
%%%%%%%%%%%%%%%%%%%%%%%%%%%%%%%%%%%%%%%%%%%%%%%%%%%%%%%%%%%%%%%%%%%
\section{Port Type Variance}
As we have seen, port types are parameterized by the types of
value that are expected to be read from (written to) them.
In contrast to Java, in which all parameterized type constructors are 
covariant in their parameter types, Scala lets us specify the variance 
of the port type constructors precisely. 
%
Below we argue that the \textsc{InPort} constructor should be covariant in its type
parameter, and the \textsc{OutPort} constructor contravariant in its type 
parameter. In other words:

\begin{enumerate}
\item If $T'$ is a subtype of $T$, then a $??[T']$ will
suffice in a context that requires a  $??[T]$; but not \textit{vice-versa}.

\item If $T'$ is a subtype of $T$, then a $!![T]$ will
suffice in a context that requires a  $!![T']$; but not \textit{vice-versa}.
   
\end{enumerate}

Our argument is, in effect,  by contradiction. To take a concrete example, suppose that
we have an interface \textsc{Printer} which has subtype
\textsc{BonjourPrinter} that has an additional method, \textsc{bonjour}.

Suppose also that we have process generators: 
\begin{code}
        def printServer  (printers: !![Printer]) : PROC = ...
        def bonjourClient(printers: ??[BonjourPrinter]) : PROC = ...
\end{code}

Then under the uniformly covariant regime of Java the following program would be
type valid, but it would be unsound:
\begin{code}
        val connector = new OneOne[BonjourPrinter]
        (printServer(connector) || bonjourClient(connector))()
\end{code}
The problem is that the server could legitimately write a non-bonjour
printer that would be of little use to a client that expects to read and use
bonjour printers. This would, of course, be trapped as a runtime
error by the JVM, but it is, surely, bad engineering practice
to rely on this lifeboat if we can avoid launching a doomed ship
in the first place!\note{This difficulty is analogous to the
well-known difficulty in Java caused by the covariance of 
the array constructor.}  
% 
And we can: for under CSO's contravariant typing of outports, the
type of \textsc{connector} is no longer a subtype of \textsc{!![Printer]},
and the expression \textsc{printServer(connector)} would, therefore,
be ill-typed.



Discussions of the variance of type constructors are easier to understand
if we think of the subtype relation as capturing ``satisfies all
the assumptions we can make about''. In what follows we write $T'\ge
T$ to mean ``a value of type $T'$ satisfies all the assumptions we
can make about a value of type $T$'', or alternatively, ``a $T'$ has
all the methods of a $T$.''

It is a general principle that if a variable $var$ has type $T$, then
it is acceptable to associate $var$ (by binding or assignment) with a 
value whose type is $\ge T$, for any \textit{uses} of $var$ in
its scope can require no more of it than is required of
a $T$, and any $T'\ge T$ provides this.

We rationalize the variances we have assigned to these constructors
as follows:
\begin{enumerate}
\item If $T'\ge T$ then $??[T']~\ge~??[T]$ 
\\Rationale: a process that reads from an input port associated with a
variable declared by $var:~??[T]$ expects to receive objects of type $\ge
T$. Thus any port of type $??[T']$ (where $T'\ge T$) can be associated
with $var$.


\item If $T'\ge T$ then $!![T]~\ge~!![T']$ 
\\Rationale: a process that writes to an output port associated with
a variable declared by $var:~!![T]$ must send objects of a type 
that is $\ge T$. Thus any port of type $!![T']$ (where $T \ge T'$) 
can be associated with $var$.

\end{enumerate}


%%%%%%%%%%%%%%%%%%%%%%%%%%%%%%%%%%%%%%%%%%%%%%%%%%%%%%%%%%%%%%%%%%%
%%%%%%%%%%%%%%%%%%%%%%%%%%%%%%%%%%%%%%%%%%%%%%%%%%%%%%%%%%%%%%%%%%%
%%%%%%%%%%%%%%%%%%%%%%%%%%%%%%%%%%%%%%%%%%%%%%%%%%%%%%%%%%%%%%%%%%%
%%%%%%%%%%%%%%%%%%%%%%%%%%%%%%%%%%%%%%%%%%%%%%%%%%%%%%%%%%%%%%%%%%%

%%%%%%%%%%%%%%%%%%%%%%%%%%%%%%%%%%%%%%%%%%%%%%%%%%%%%%%%%%%
%%%%%%%%%%%%%%%%%%%%%%%%%%%%%%%%%%%%%%%%%%%%%%%%%%%%%%%%%%%
%%%%%%%%%%%%%%%%%%%%%%%%%%%%%%%%%%%%%%%%%%%%%%%%%%%%%%%%%%%
%%%%%%%%%%%%%%%%%%%%%%%%%%%%%%%%%%%%%%%%%%%%%%%%%%%%%%%%%%%


%%%%%%%%%%%%%%%%%%%%%%%%%%%%%%%%%%%%%%%%%%%%%%%%%%%%%%%%%%%%%%%%%%%%%%%%%%%%%%%%%%%
%%%%%%%%%%%%%%%%%%%%%%%%%%%%%%%%%%%%%%%%%%%%%%%%%%%%%%%%%%%%%%%%%%%%%%%%%%%%%%%%%%%
%%%%%%%%%%%%%%%%%%%%%%%%%%%%%%%%%%%%%%%%%%%%%%%%%%%%%%%%%%%%%%%%%%%%%%%%%%%%%%%%%%%
\section{Prospects}

We remain committed to the challenge of developing Scala+CSO both
as a pedagogical tool and in the implementation of realistic
and efficient programs.

Several small-scale and a few medium-scale case studies on networked
multicore machines have given us some confidence that our implementation
is sound, though we have neither proofs of this nor a body of
successful (\textit{i.e.} non-failed) model checks. The techniques
pioneered by Welch and Martin in \cite{welchandmartin} show the way
this could be done.

In an earlier version of this paper, we wrote:
\begin{itemize}\item[]
A few years ago Andrew Bate implemented a very high performance
variant of CSO, in which huge numbers of running processes can be
multiplexed (as ``fibres'') across smaller numbers of threads. We
would very much like to make Andrew's dialect and the dialect
described here compatible at the source-code level, but Andrew's
implementation depended on extensive post-processing of the JVM
code generated by Scala/Java, and keeping it up to date would require
the evolution of the Scala compiler to be tracked: something we 
don't have the resources to commit to.
\end{itemize}

We are delighted to say that the advent of \textbf{virtual threads}
in the JVM (in jdk 20 and subsequent) meant that our implementation
was easily adapted to delivering ``huge numbers of running processes''
very efficiently. The API explains the details, but the short story
is that \textit{processes will now run as virtual threads by default
on a JVM that implements virtual threads}.


\section{Distributed Programming with CSO}
The prototype  CSO library \textbf{eieio} (Extensible Interface to External I/O)
provides components that can be used to implement distributed and 
networked programs. It maps external sockets to internal channels and handles 
the details of serializing and deserialising data. 




%%%%%%%%%%%%%%%%%%%%%%%%%%%%%%%%%%%%%%%%%%%%%%%%%%%%%%%%%%%%%%%%%%%%%%%%%%%%%%%%%%%
%%%%%%%%%%%%%%%%%%%%%%%%%%%%%%%%%%%%%%%%%%%%%%%%%%%%%%%%%%%%%%%%%%%%%%%%%%%%%%%%%%%
%%%%%%%%%%%%%%%%%%%%%%%%%%%%%%%%%%%%%%%%%%%%%%%%%%%%%%%%%%%%%%%%%%%%%%%%%%%%%%%%%%%
\newpage
\section*{Acknowledgements}
We are grateful to Peter Welch, Gerald Hilderink and their
collaborators whose early demonstration of the feasibility of using
\occam-like concurrency in a Java-like language inspired us in the first place.
Also to Martin Odersky and his collaborators in the design and
implementation of the Scala language.

Several of our students on the Concurrent and Distributed Programming
course at Oxford helped us by testing the present work and by
questioning the work that preceded and precipitated it.  Itay Neeman
drew Scala to our attention and participated in the implementation
of a prototype of CSO. Dalizo Zuse and Xie He helped us refine our
ideas by building families of Web-servers -- using a JCSP-like
library and the CSO prototype respectively.

Our colleagues Michael Goldsmith and Gavin Lowe (and later
Andrew Bate) were a constant source of technical advice and
friendly skepticism; and Quentin Miller joined us in research that
led to the design and prototype implementation of ECSP \cite{ecsp,ecspman}.

M. Ahsan al Mahir, of St John's College Oxford, was responsible in
February 2023 for the first implementation of the virtual thread
gactory that was needed to enable processes to run as virtual
threads. He was also responsible for adapting the source repository
and making it compatible with \texttt{sbt}.

Last, but not least, we are grateful to Carsten Heinz and his
collaborators for their powerful and versatile LaTeX \texttt{listings} package
\cite{listings}.

%%%%%%%%%%%%%%%%%%%%%%%%%%%%%%%%%%%%%%%%%%%%%%%%%%%%%%%%%%%%%%%%%%%%%%%%%%%%%%%%%%%
%%%%%%%%%%%%%%%%%%%%%%%%%%%%%%%%%%%%%%%%%%%%%%%%%%%%%%%%%%%%%%%%%%%%%%%%%%%%%%%%%%%
%%%%%%%%%%%%%%%%%%%%%%%%%%%%%%%%%%%%%%%%%%%%%%%%%%%%%%%%%%%%%%%%%%%%%%%%%%%%%%%%%%%
\newpage\section*{Appendix: Thumbnail Scala and the Coding of CSO}
In many respects Scala is a conventional object oriented language
semantically very similar to Java, though notationally somewhat
different.\note{The main distributed Scala
implementation translates directly into the JVM; though another compiler
translates into the \texttt{.net} CLR. The existence of
the latter compiler encouraged us to build a pure Scala
CSO library rather than simply providing wrappers for the 
longer-established JCSP library.}
%
It has a number of features that have led some to describe it as a
\textit{hybrid} functional and object-oriented language, notably
\begin{itemize}
\item \textit{Case classes} make it
      easy to represent free datatypes and to program with them.

\item \textit{Functions are first-class values}. The type expression \textsc{T=>U} 
      denotes the type of functions that map values of type \textsc{T}
      into values of type \textsc{U}. One way of denoting
      such a function anonymously is \textsc{\{ bv => body \}}
      (providing \textsc{body} has type \textsc{U}).\note{In
      some contexts fuller type information has to be given,
      as in: \textsc{\{ case bv: T => body \}}. Functions 
      may also be defined by cases over free types; for an example see 
      the match expression within \textsc{receiver} in section \ref{Timed Alternation}}
\end{itemize}
The principal novel features of Scala we used in making
CSO  notationally palatable were:
\begin{itemize}
\item   \textit{Syntactic extensibility}: objects may have methods
        whose names are symbolic operators; and an object
        with an \textsc{apply} method may be ``applied'' to
        an argument as if it were a function.

\item   \textit{Call by Name}: a Scala function or method may have 
        have one or more parameters of type \textsc{=> T}, in which
        case they are given ``call by name'' semantics and the
        actual parameter expression is evaluated anew whenever the
        formal parameter name is mentioned.

\item   \textit{Code blocks}: an expression of the form $\{...\}$
        may appear as the actual parameter corresponding to a formal
        parameter of type \textsc{=> T}. 
\end{itemize}
The following extracts from the CSO implementation show
these features used in the implementation of unguarded repetition and \textsc{proc}.
\begin{code}
// Implementing unguarded repetition
def repeat (cmd: => Unit) : Unit =
{ var go = true
  while (go) 
    try   { cmd } 
    catch { case io.threadcso.process.Stopped(_,_) => go=false } 
}

// Definition of proc syntax
def proc (body: => Unit) : PROC = 
    new Process.Simple(()=>body).withName(Process.genName)
\end{code}
Implementation of the guarded event notation of section \ref{Input Alternation}
is more complex. For example, the formation of an 
input event from the 
Scala expression  $(guard\mathtt{\&\&}port)\mathtt{=?\!\!=>}~rhs$ 
takes place in two stages: first the evaluation of $(guard\mathtt{\&\&}port)$ yields an
intermediate \textsc{GuardedInPort} object, $ev$; then the evaluation of
$ev\mathtt{=?\!\!=>}~rhs$ yields the \textsc{InPortEvent} that will be
a candidate for selection and execution. An unguarded event
is constructed as an \textsc{InPortEvent} in a simple step.


%%%%%%%%%%%%%%%%%%%%%%%%%%%%%%%%%%%%%%%%%%%%%%%%%%%%%%%%%%%%%%%%%%%%%%%%%%%%%%%%%%%
%%%%%%%%%%%%%%%%%%%%%%%%%%%%%%%%%%%%%%%%%%%%%%%%%%%%%%%%%%%%%%%%%%%%%%%%%%%%%%%%%%%
%%%%%%%%%%%%%%%%%%%%%%%%%%%%%%%%%%%%%%%%%%%%%%%%%%%%%%%%%%%%%%%%%%%%%%%%%%%%%%%%%%%
\newpage\section*{Appendix: JCSP Fair Multiplexer}
Program \ref{javafairplex} shows the JCSP implementation of a fair multiplexer
component (taken from \cite{jcsp}) for comparison with the CSO implementation
of the component with the same functionality in section \ref{scalafairplex}.
\codestyle{java}
\begin{code+}[...]{frame=lrtb,label={javafairplex},caption=Fair Multiplexer Component using JCSP}
  public final class FairPlex implements CSProcess { 
      private final AltingChannelInput[] in; 
      private final ChannelOutput        out;
      public FairPlex ( AltingChannelInput[] in, ChannelOutput out) 
      { this.in = in; this.out = out; }
       
      public void run () { 
          final Alternative alt = new Alternative (in); 
          while (true) {  final int i = alt.fairSelect (); 
                          out.write (in[i].read ()); 
          } 
      } 
  } 
\end{code+}
\codestyle{scala}



%%%%%%%%%%%%%%%%%%%%%%%%%% NOTES

\listofendnotes

%%%%%%%%%%%%%%%%%%%%%%%%%%
\def\href#1{{\def\~{$\sim$}\ttfamily\footnotesize #1}}

%\newpage
\begin{thebibliography}{99}


\bibitem{ctj}
Hilderink G.,
Broenink J.,
Vervoort W. and
Bakkers A.
Communicating Java Threads.
In \textit{Proceedings of WoTUG-20: ISBN 90 5199 336 6} (1997) 48--76.

\bibitem{ctjapp}
Hilderink G., Bakkers A. and Broenink J.
A Distributed Real-Time Java System Based on CSP.
In \textit{Proceedings of the Third IEEE International Symposium on 
Object-Oriented Real-Time Distributed Computing}, (2000), 400--407. 

\bibitem{jcsp}
Website for Communicating Sequential Processes for Java, (2008)
\\\href{http://www.cs.kent.ac.uk/projects/ofa/jcsp/}

\bibitem{jcsp2}
Welch P. \textit{et. al.} Integrating and Extending JCSP. 
In \textit{Communicating Process Architectures} (2007) 48--76.

\bibitem{ecsp}
Miller Q. and Sufrin B. Eclectic CSP: a language of concurrent processes. 
In \textit{Proceedings of 2000 ACM symposium on Applied computing}, (2000) 840--842. 

\bibitem{ecspman}
Sufrin B. and Miller Q. Eclectic CSP. \textit{OUCL Technical Report}, (1998)
\\\href{http://users.comlab.ox.ac.uk/bernard.sufrin/ECSP/ecsp.pdf}

\bibitem{dearsir}
Welch P. \textit{et. al.} Letter to the Editor. \textit{IEEE Computer}, (1997)
\\\href{http://www.cs.bris.ac.uk/\~alan/Java/ieeelet.html}

\bibitem{jfc1}
Muller H. and Walrath K. Threads and Swing. \textit{Sun Developer Network}, (2000)
\\\href{http://java.sun.com/products/jfc/tsc/articles/threads/threads1.html}


\bibitem{scala}
Odersky M. \textit{et. al.} An Overview of the Scala Programming Language.
\textit{Technical Report LAMP-REPORT-2006-001, EPFL, 1015 Lausanne, Switzerland} (2006)
(also at \href{http://www.scala-lang.org/})

\bibitem{welchandmartin}
Welch P. and Martin J. Formal Analysis of Concurrent Java Systems.
In \textit{Proceedings of CPA 2000 (WoTUG-23): ISBN 1 58603 077 9} (2000) 275--301.


\bibitem{jcspnet}
Welch P. CSP Networking for Java (JCSP.net)
\\\href{http://www.cs.kent.ac.uk/projects/ofa/jcsp/jcsp-net-slides-6up.pdf}


\bibitem{actors1}
Haller P. and Odersky M. Event-based Programming without Inversion of Control.
In: Lightfoot D.E. and Szyperski C.A. (Eds) JMLC 2006. LNCS 4228, 4--22. 
Springer-Verlag, Heidelberg (2006)
\\(also at \href{http://www.scala-lang.org/})

\bibitem{BGJK}
Geoff Barrett, Michael Goldsmith, Geraint Jones, and Andrew Kay.
The meaning and implementation of PRI ALT in
\textbf{\textsf{occam}}.  In \textbf{\textsf{occam}} and the Transputer, 
research and applications (OUG-9), ed. Charlie Askew, IOS, (September 1988).
\\(also at \href{ftp://ftp.comlab.ox.ac.uk/pub/Documents/techpapers/Geraint.Jones/\\OCCAM-1-88.ps.Z})

\bibitem{actors2}
Haller P. and Odersky M. Actors that unify Threads and Events.
In: Murphy A.L. and Vitek J. (Eds) COORDINATION 2007. LNCS 4467, 171--190.
Springer-Verlag, Heidelberg (2006)
\\(also at \href{http://www.scala-lang.org/})

\bibitem{listings}
Heinz C. The \texttt{\textbf{listings}} Package.
\\(\href{http://www.ifi.uio.no/it/latex-links/listings-1.3.pdf})

\bibitem{lowe}
Lowe, G. Implementing Generalised Alt. 
In \textit{Proceedings of Concurrent Process Architectures} (2011) \mbox{1--34}. 
\\(\href{http://www.cs.ox.ac.uk/gavin.lowe/Papers/alt.pdf})

\end{thebibliography}

%\appendix
%\section*{Scala}
%\section*{CSO}

%\begin{comment}
%\newpage
\vfill
\vfill
\vfill

{\footnotesize (\verb$Id: cso-paper.tex 306 2017-12-19 16:29:29Z sufrin $)}

\vfill
%\end{comment}
\end{document}


























