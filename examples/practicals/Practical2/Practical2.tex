\documentclass{../csopractical}
\SVN    $Id: Practical2.tex 1324 2016-01-15 18:19:56Z sufrin $

\begin{document}
\begin{center}
\Large\bf Concurrent Programming Practical 
\\
2: The Game of Life 
\end{center}

It is recommended that you read the chapter of the coursenotes on synchronous
data parallel programming before you attempt this practical.  The code from
the lectures is on the course website.

\subsection*{Introduction}

The aim of this practical is to implement the Game of Life, a cellular
automaton invented by John  Conway in 1970.  The Wikipedia entry on Life
describes it as
follows:\footnote{\url{http://en.wikipedia.org/wiki/Conway's_Game_of_Life}} 
%
\begin{quote}
The universe of the Game of Life is an infinite two-dimensional orthogonal
grid of square cells, each of which is in one of two possible states, live or
dead. Every cell interacts with its eight neighbours, which are the cells that
are directly horizontally, vertically, or diagonally adjacent. At each step in
time, the following transitions occur: 
%
\begin{enumerate}
\item
Any live cell with fewer than two live neighbours dies, as if by needs caused
by underpopulation.

\item
Any live cell with more than three live neighbours dies, as if by overcrowding.

\item
Any live cell with two or three live neighbours lives, unchanged, to the next
generation. 

\item
Any tile with exactly three live neighbours cells will be populated with a
living cell.
\end{enumerate}

The initial pattern constitutes the `seed' of the system. The first generation
is created by applying the above rules simultaneously to every cell in the
seed --- births and deaths happen simultaneously, and the discrete moment at
which this happens is sometimes called a tick. (In other words, each
generation is a pure function of the one before.) The rules continue to be
applied repeatedly to create further generations. 
\end{quote}
%
You might want to read the rest of the Wikipedia article.

We will consider a variant with a finite $N$ by $N$ grid, treated as a toroid,
i.e.\ where the top and bottom edges of the grid are treated as being
adjacent, as are the left and right edges.

%%%%%

\subsection*{Your task}

Your task is to implement the Game of Life.  Your program should
use $p$ processes (say 4 or 8) to update the cells on each generation.
You will probably want to allocate some region of the grid to each
process.  There are two essentially different ways to structure the
program: (1)~as a shared-variable synchronous data parallel program;
(2)~as a pure message-passing program.\footnote {In the latter case
my experience with a predecessor of thread CSO suggests that
allocating a CSO process to each square in the grid means that only
a small grid can be dealt with before the operating system/jvm
begins to run out of thread resources. Go, or the fibre-based
implementation of CSO would be a better bet for a serious attempt:
but my advice is not to try at this stage. Other message-passing
solutions have workers working on larger rectangular regions of the grid,
and exchanging the edges of their regions with neighbour workers
every generation.}  In both cases the processes need to be synchronised
in some way, so that the rules of the game are followed, so you need
to think carefully about how to avoid race conditions.

There is a file
\texttt{Display.scala} (on the course website), which defines
a \texttt{Display} class that can be used to build displays
that show the state of a grid.  A
display is created by
%
\begin{scala}
  val display = new Display(N,a)
\end{scala}
%
where \SCALA{N} is an integer, representing the size of the grid, and
\SCALA{a} is the grid, represented by an \SCALA{N} by \SCALA{N} array of \SCALA{Boolean},
e.g.\ initialised by
%
\begin{scala}
  val a =  Array.ofDim[Boolean](N,N)
\end{scala}
%
The display is made consistent with \SCALA{a} by executing
%
\begin{scala}
  display.draw
\end{scala}
%
This shows solid black squares at positions \SCALA{(i,j)} where \SCALA{a(i,j)} is true,
and grey squares where \SCALA{a(i,j)} is false.

%% You might also want to use the file {\tt Barrier.scala}, which provides an
%% implementation of barrier synchronisation.

Test your implementation by considering some interesting seeds (see the
Wikipedia page to get some ideas).

%%%%%

\subsection*{Optional}

Implement one or more of the variants from the Wikipedia page.  (You might
have to adapt {\tt Display.scala} to do this.)
%
Alternatively, devise a more complicated game that can be modelled using
cellular automata.  For example, you could model the interactions between
foxes and rabbits.  Don't make the rules too complicated, though!  You could
include some randomness in the game.  

%%%%%

\subsection*{Reporting}

Your report should be in the form of a well commented program, describing any
design decisions you have made.

%% \begin{flushright}
%% Gavin Lowe \\ February 2010
%% \end{flushright}

\end{document}








