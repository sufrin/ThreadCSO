\documentclass{../csopractical}
\SVN    $Id: Practical4.tex 1571 2017-10-16 14:14:21Z sufrin $
\def\Var#1{\textbf{#1}\newline}
\begin{document}
\begin{center}
\Large\bf Concurrent Programming Practical 
\\
4: Dining Philosophers
\end{center}
%\maketitle

It is recommended that you read the material from the section of the
course notes on the Dining Philosophers before you attempt this practical.  The
code from the lectures is on the course website.
The aim of the practical is to investigate some variants of the Dining
Philosophers, which aim to avoid deadlocks.  You should implement the
first three variants below, and as many of the optional variants you
have the inclination to work on. 

%%%%%

\Var{Variant 1: a right-handed philosopher}
In the standard version of the dining philosophers, all the philosophers are
left-handed: they pick up their left fork first.  Implement a variant where
one of the philosophers is right-handed, i.e.\ she picks up her right fork
first. 

%%%%%

\Var{Variant 2: using a butler}
Now consider a variant using an extra process, which represents a butler.  The
butler makes sure that no more than four philosophers are ever simultaneously
seated. 

%%%%%

\Var{Variant 3: asking the forks using a protocol}
Now consider a variant where, if a philosopher is unable to obtain her second
fork, she puts down her first fork, and re-tries later.  
This can be solved straighforwardly by implementing a bidirectional
protocol between forks and philosophers: a philosopher should be able ask
a fork to commit itself to her, and be answered positively only if the fork has actually 
committed to her; negatively if the fork has already committed to its other 
potential user. [Hint: the protocol will need two channels; it will simplify
your program structure to package them in a class.]

\Var{Optional, but desirable} For each of the variants you implement include a way of
counting the number of times each philosopher has eaten, and of presenting all this
information periodically in addition to, or instead of, the detailed reports
of the actions of forks and philosophers. Think of this as insurance against 
starvation! The same code should suffice for all variants.

%%%%%%

\Var{Variant (optional: for fun): pile the forks in the middle of the table} 
In this variant, forks that are not being used sit in a pile in the middle of
the table, and philosophers can use any two of them to eat.

\Var{Variant (optional: timeouts)} Now consider a
variant where, if a philosopher is unable to obtain her
second fork, she puts down her first fork, and re-tries
later. Hint: This can be solved straighforwardly by using
the deadlined `writeBefore` method in the channel from a
philosopher to one of her forks.\footnote{It may be worth doing some
simple experiments in advance to see how deadlined writes/reads interact 
with alternations. My own solution used a deadline that was roughly twice 
the eating time.} 


%%%%%

\Var{Reporting}

Your report should be in the form of a well-commented program, together with
a brief discussion of any design considerations and of your results.

\end{document}





















