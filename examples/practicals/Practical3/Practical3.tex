\documentclass{../csopractical}
\usepackage{pdffig}
\SVN    $Id: Practical3.tex 1325 2016-01-15 19:11:41Z sufrin $

\begin{document}
\begin{center}
\Large\bf Concurrent Programming Practical 
\\
3: Hamming Numbers 
\end{center}
%\maketitle

It is recommended that you read the material from the section of the course on
\emph{Message Passing} (using Channels) before you attempt this practical.

The aim of the practical is to  implement a dataflow network to generate
an ordered sequence of \textit{Hamming Numbers}, with no duplicates.
The (inductive) definition of the Hamming numbers is as follows:
%
\begin{itemize}
\item 1 is a Hamming number.
\item If $h$ is a Hamming number, then so are $2h$, $3h$ and $5h$.
\end{itemize}
%
Another way of thinking of the Hamming numbers is that they are those whose
only prime factors are 2, 3 and 5.

The illustration below shows a network that solves the problem. If you
have read the material suggested, it should be fairly obvious what the
various components are expected to do.  The component \SCALA{Merge} receives
two ordered streams, and merges them into a single ordered stream, without
repetitions. 

\begin{center}
\pdffig[width=6cm]{hamming}
\end{center}

\subsubsection*{Your Task}
Implement the network.  

You may make use of components from the package
\SCALA{io.threadcso.component} if you like (but note that
\SCALA{io.threadcso.component.merge} has a different behaviour from the
\SCALA{Merge} required here).  In some cases you will need to
implement your own components, especially to answer some of the questions
below. 

If you implement the network with no buffering on any of the channels, you
will find that it deadlocks.  Explain precisely how this deadlock
arises: you should explain what state each component is in in the deadlocked
state.  You might need to adapt some of the components in order to understand
this behaviour.

Re-implement the network with sufficient buffering to enable it to find the value
of the 1000th Hamming number.  The declaration
\begin{scala}
val c = OneOneBuf[T](size)
\end{scala}
defines \SCALA{c} to be an asynchronous buffered channel, passing data of
type~\SCALA{T}, with capacity~\SCALA{size}.  Think about which channels
need to be buffered, and which can remain unbuffered.

{\bf Optional:} Modify the components so that all the processes terminate
cleanly once the 1000th Hamming number has been output.

\textbf{Just for fun:}
Generalize the construction of the network so that it generates
the sequence of ``Hamming-like'' numbers whose  factors are given as a 
list of parameters to your program.


\subsubsection*{Reporting}

Your report should be in the form of a well-commented program, together with
brief answers to the above points. 

%% You have already seen the design of a \texttt{Tee}-like
%% component. You are expected to design the remaining
%% components (\texttt{Merge, Buf, x2, x3, x5}) that will
%% enable you to explore the behaviour of the
%% network but you must decide for yourself how to
%% ``bootstrap'' the data flow in the network, and how and
%% where to ``tap'' the network so that the numbers can be
%% output to a channel from which they can be read and used.
%% In the diagram these details are hidden within the component named
%% \texttt{Hamming}.

%% Your report (which need not be very long) should explain your
%% code. In it you should also document the results of your investigation into the
%% following questions:
%% \begin{enumerate}\item[]
%% \begin{enumerate}\item[]
%% \begin{enumerate}
%% \item Does the network deadlock? 
%% \item If so, then under what circumstances does it deadlock?
%% \item Could the deadlock either be avoided or postponed by changing the 
%%       design of the network or by increasing the size of one or more
%%       of the buffers?
%% \item What is the largest Hamming number representable as a \SCALA{long}?
%% \item What is the 10000'th Hamming number?
%% \item Can any network component be modified, or can
%%       suitable components be added to the network, to ensure that all
%%       processes involved terminate cleanly after the largest representable
%%       Hamming number has been output?
%% \end{enumerate}
%% \end{enumerate}
%% \end{enumerate}

\vfill


\end{document}


















